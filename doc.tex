\documentclass[a4paper,12pt]{article}
\usepackage[utf8]{inputenc}
\usepackage[brazil]{babel}
\usepackage{amsmath, amssymb}
\usepackage{graphicx}
\usepackage{listings}
\usepackage{color}
\usepackage{hyperref}

\title{Trabalho Prático 1 - Comunicação Cliente/Servidor com TCP}
\author{Seu Nome - Matrícula: XXXXXXXX}
\date{\today}

\begin{document}

\maketitle

\section*{1. Introdução}
Este trabalho prático consiste na implementação de um jogo multiplayer baseado no conceito de Jokenpô estendido, chamado Jokenboom, utilizando comunicação via sockets TCP. O jogo é implementado na linguagem C, com suporte tanto para IPv4 quanto IPv6, usando a API de sockets POSIX.

\section*{2. Organização do Projeto}
O projeto está estruturado da seguinte forma:
\begin{itemize}
    \item \textbf{client.c} -- responsável pela comunicação com o servidor e interação com o jogador.
    \item \textbf{server.c} -- gerencia a lógica do jogo, sorteia as jogadas do servidor, avalia os resultados e gerencia o placar.
    \item \textbf{common.c/h} -- funções utilitárias para manipulação de endereços IP e sockets.
    \item \textbf{protocol.c/h} -- define a estrutura \texttt{GameMessage} e funções auxiliares para inicialização e depuração.
    \item \textbf{Makefile} -- compila os binários \texttt{bin/client} e \texttt{bin/server}.
\end{itemize}

\section*{3. Detalhes da Implementação}

\subsection*{3.1 Estrutura GameMessage}
A estrutura \texttt{GameMessage} representa as mensagens trocadas entre cliente e servidor. Ela contém campos como tipo da mensagem, jogadas, resultado e placar:

\begin{lstlisting}[language=C, basicstyle=\ttfamily\footnotesize]
typedef struct {
    int type;
    int client_action;
    int server_action;
    int result;
    int client_wins;
    int server_wins;
    char message[256];
} GameMessage;
\end{lstlisting}

\subsection*{3.2 Lógica do Servidor}
O servidor aguarda conexões TCP, envia solicitações de jogada ao cliente, recebe respostas e decide o resultado com base em uma matriz de regras:

\begin{lstlisting}[language=C]
int tabela[5][5] = {
    { -1, 0, 1, 1, 0 },
    {  1, -1, 1, 1, 0 },
    {  0, 0, -1, 1, 1 },
    {  0, 0, 0, -1, 1 },
    {  1, 1, 0, 0, -1 }
};
\end{lstlisting}

Essa tabela representa o resultado entre cada par de jogadas possíveis.

\subsection*{3.3 Protocolo de Comunicação}
\begin{enumerate}
    \item \texttt{MSG\_REQUEST}: servidor solicita jogada ao cliente.
    \item \texttt{MSG\_RESPONSE}: cliente envia a jogada.
    \item \texttt{MSG\_RESULT}: servidor envia o resultado da rodada.
    \item \texttt{MSG\_PLAY\_AGAIN\_REQUEST}: servidor pergunta se o cliente deseja continuar.
    \item \texttt{MSG\_PLAY\_AGAIN\_RESPONSE}: cliente responde se quer continuar.
    \item \texttt{MSG\_END}: servidor envia o placar final e encerra a conexão.
    \item \texttt{MSG\_ERROR}: servidor envia mensagem de erro se a entrada do cliente for inválida.
\end{enumerate}

\section*{4. Compilação e Execução}
Para compilar o projeto, execute:
\begin{verbatim}
make
\end{verbatim}
Os binários serão gerados na pasta \texttt{bin/}. Para executar:

Servidor:
\begin{verbatim}
./bin/server v4 51511
\end{verbatim}

Cliente:
\begin{verbatim}
./bin/client 127.0.0.1 51511
\end{verbatim}

\section*{5. Conclusão}
O projeto permitiu aplicar conceitos fundamentais de redes de computadores, como comunicação via sockets TCP, tratamento de mensagens estruturadas, e suporte a múltiplas famílias de protocolo IP. A modularização do código facilitou a manutenção e a clareza na implementação.

\end{document}
